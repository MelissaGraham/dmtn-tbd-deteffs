\documentclass[DM,lsstdraft,toc]{lsstdoc}
\usepackage{graphicx}
\usepackage{url}
\usepackage{latexsym}
\usepackage{color}
% black, blue, brown, cyan, darkgray, gray, green, lightgray, lime, magenta, blue, orange, pink, purple, red, teal, violet, white, yellow.
\usepackage{enumitem}
\usepackage{esvect}

\title[Detection Efficiencies]{Characterizing {\tt DIASource} Detection Efficiency}

\author{M.~L.~Graham et al., and the DM SST}

\setDocRef{DMTN-TBD}
\date{\today}
\setDocUpstreamLocation{\url{https://github.com/lsst-dm/dmtn-tbd}}


% % % % % % % % % % % % % % % % % %
\setDocAbstract{The goals of this study are to: (1) validate that the planned DM data products which characterize {\tt DIASource} detectability will meet the scientific needs of the community; (2) ensure that the requirements related to the characterization of {\tt DIASource} detectability are adequately flowed down, and that DM's plans are accurately expressed in the DM documentation; and (3) inform the community about the planned DM data products related to {\tt DIASource} detectability.}


\setDocChangeRecord{%
\addtohist{0.0}{2018-11-01}{Internal working document.}{Melissa Graham}
\addtohist{0.1}{2019-09-09}{Updated to represent DM's plans.}{Melissa Graham}
\addtohist{1}{2022-04-dd}{Update and release.}{Melissa Graham}
}

\begin{document}

\maketitle

% CITATION EXAMPLES
% \verb|\citellp|: \citellp{LPM-17, LSE-30} \\
% \verb|\citell|: (SRD; \citell{LPM-17,LSE-29}) \\
% \verb|\citep[][]|: \citep[e.g.,][are interesting]{LPM-17,LSE-29} \\
% \verb|\cite|: \cite{LPM-17,LSE-29}
% \citeds{LSE-163}, \citedsp{LSE-163}


% % % % % % % % % % % % % % % % % %
\section{Introduction} \label{sec:intro}

Any astrophysical question which asks, e.g., {\it "How often?"} or {\it "How many?"} about transient phenomena, such as population studies or occurrence rates, needs to know a survey's {\it detection efficiency}: the probability that a source is detected (or in other words, the fraction of all sources that are detected).

Detection efficiencies are required for many of the core science pillars of the LSST, such as transient phenomena, cosmology, and Solar System studies; many other surveys use populations of synthetic sources injected into the data in order to characterize the detection efficiency (Appendix~\ref{sec:sci}).
This should be done for the LSST as well -- deriving detection efficiencies is too large and complicated a task to leave for the science community.

Rubin Observatory documentation contains no requirements related to producing or serving detection efficiencies for transient {\tt DIASources} in difference images (Appendix~\ref{sec:docs}), but does contain requirements about using source injection to characterize detected source {\it spuriousness} (see also Appendix~\ref{sec:rb}).

This document describes how Rubin Observatory should use its synthetic source injection algorithms to measure and provide detection efficiencies for transient point sources detected by the LSST via difference image analysis (DIA).

Readers should refer to the Data Products Definitions Document \citedsp{LSE-163} for more information about DIA, difference images, and {\tt DIASources}.
Detection efficiencies of extended difference-image sources (e.g., light echoes, trailed moving objects), or for variable or static point-sources in direct images, are beyond the scope of this document.


\section{Proposed Detection Efficiency Matrix}

{\bf Detection:} as described in the SRD and DPDD, sources in difference images with a signal-to-noise ratio $SNR > {transSNR} = 5$ will be considered {\it detected} and will become a {\tt DIASource}.
 
{\bf Detection Efficiency:} The probability that a true astrophysical source of a given magnitude is detected and becomes a {\tt DIASource}. In other words, the fraction of all true astrophysical sources of a given magnitude that are detected and become {\tt DIASources}. 

Characterizing detection efficiency requires {\it knowing} which of the detected {\tt DIAsources} are astrophysical, and the total number of astrophysical sources in the field of the image.
This would only be possible if ``truth" is somehow known from, e.g., co-temporal imaging data of superior quality -- but of course that almost never exists because it is inefficient to duplicate data.

Instead, the detection efficiency is typically characterized by simulating and injecting synthetic sources into images, using a point-spread function shape like real astrophysical sources, and then processing the images with the survey's difference image analysis pipeline and measuring the fraction recovered.

The detection efficiency can be characterized as $\eta(m)$, where $m$ is the apparent magnitude of the time-changing component with respect to the template image, and $\eta$ is a value between $0$ and $1$ that represents the probability that the source would be detected in the difference image.
As described in Appendix~\ref{sec:sci}, $\eta$ depends on more than just $m$, and is a function of the parameters ($\vv{P}$) listed in Table \ref{tab:eta_pars}.

An accurate measure of $\eta(m,\vv{P})$ for every pixel of every difference image is technologically unfeasible.
Instead, an analytic model for $\eta$ as a function of both $m$ and parameters $\vv{P}$ (some of which may be correlated) can be built with synthetic source injection, and then applied by users.

This means that the synthetic sources should be simulated over a range of parameters $\vv{P}$, but as described in Appendix~\ref{sec:sci} the simulated sources do not need to accurately represent astrophysical transient types, colors, redshifts, durations, light curves, etc.
That aspect of the analysis is best left to the user to handle during the MC simulation stage for their particular type of object.
 
\begin{table}[h]
\begin{center}
\begin{footnotesize}
\caption[]{A description of the image and source parameters ($\vv{P}$) that can affect the detection efficiency ($\eta$) of point sources in a difference image (per filter).}
\label{tab:eta_pars}
\setlength{\extrarowheight}{5pt}
\begin{tabular}{|p{3.1cm}|p{12cm}|}
\hline
{\bf Parameter} & {\bf Description} \\
\hline
%Apparent Magnitude & Typically, $\eta$ decreases for fainter objects (and brighter objects due to saturation). \\
%\hline
Surface Brightness & Typically, $\eta$ decreases for objects embedded in brighter host galaxies. \\
\hline
Static-Source Offset & Sometimes, $\eta$ decreases for objects that are near (i.e., overlap the point-spread function of) static sources (e.g., stars, galaxy cores, especially if cuspy in profile). \\
\hline
CCD Location & With some instruments, $\eta$ decreases near the CCD edges due to distortion. \\
\hline
Image FWHM & The value of $\eta$ can decrease for extreme FWHM differences from the template (i.e., very good or very poor seeing). \\
\hline
Image Airmass & LSST images will experience differential chromatic refraction which affects image subtraction \citedsp{DMTN-037}, and thus potentially also $\eta$. \\
\hline
Sky Brightness & Typically, $\eta$ decreases when the sky background is bright or has a strong gradient (e.g., during twilight, near the moon). \\
\hline
Sky Cloud Cover & Extinction will affect $\eta$ by degrading the image magnitude limit. \\
\hline
\end{tabular}
\end{footnotesize}
\end{center}
\end{table}


%\subsection{Study Overview}
%\begin{itemize}
%\item \S~\ref{sec:sci} contains a primer on the concepts related to detection efficiencies, and an overview of several science goals that rely on them, with examples of methods for generating and applying detection efficiencies from a selection of recent time-domain surveys. 
%\item \S~\ref{sec:docs} reviews the LSST Data Management System's (DMS) literature and highlights existing plans, policies, requirements, and specifications related to the generation of LSST survey detection efficiencies. 
%\item \S~\ref{sec:opts} presents and evaluates various options for how the DMS could provide detection efficiencies as a data product. 
%\item \S~\ref{sec:tech} outlines several techniques for injecting fake point source variability into images.
%\end{itemize}

%\subsection{Summary of Study Findings}
%\begin{enumerate}
%\item It would enable some science -- and require no increase in scope -- to make available (e.g., as part of the annual DR) the data for the simulated point sources that will be used to characterize the {\tt DIASouce} spuriousness parameter in order to meet requirements in \citedsp{LSE-61}. This will enable users to build detection efficiency matrices, $\eta(\vv{P})$, for their science goals.
%\item It would enable more science -- and require only a small increase in scope -- to ensure that the fake point sources cover the parameter space needed for scientific analyses involving detection efficiencies, which may be different from that of internal uses. 
%\item It would be more scientifically useful -- again with another small increase in scope -- to calculate and provide the detection efficiency matrix itself, $\eta(\vv{P})$, as a data product (e.g., in the annual DR).
%\item Further study (simulations) will be needed to identify the full set of parameters $\vv{P}$ for which $\eta$ should be determined, and optimal mode of fake injection (e.g., modeling the point spread function or using isolated stars).
%\item Soliciting feedback and science use-cases from the community should be considered as a means to prioritize and make decisions regarding the above.
%\end{enumerate}

%See also \S~\ref{sec:future}.


\clearpage
\bibliography{local,lsst,refs,books,refs_ads}

% % % % % % % % % % % % % % % % % %
\clearpage
\appendix

\input{ap_sci.tex}

\input{ap_docs.tex}

\input{ap_rb.tex}

\section{Options Regarding Detection Efficiencies} \label{sec:opts}

The options for DM to participate in generating detection efficiencies, $\eta(\vv{P})$, are listed and discussed in terms of scope, risk, requirements, and science. 

% % % % % % % % % % % % % % % % 
\subsection{Do Nothing}\label{ssec:opts_no}

In this scenario, the data from any fake injection that is done in order to meet the requirement to characterize the spuriousness is not made available, but the science community would have access to the {\it software} for fake injection.

{\bf Scope --} No expansion of scope. \\
{\bf Risk --} A moderate risk in that multiple user groups may then need to perform fake injection, leading to redundant reprocessing of the data and a computational strain on the resources. \\
{\bf Requirements --} Does not violate or fulfill any requirements. \\
{\bf Science --} This option would negatively impact science results based on transient phenomena, one of the four pillars of LSST science. The need for computationally intensive processing would force multiple teams to compete, and might limit the number of individuals or teams who could successfully derive detection efficiencies, and thus limit the scientific applications.

% % % % % % % % % % % % % % % % 
\subsection{Make Available the Fakes Injected for Spuriousness Characterization}\label{ssec:opts_makefakeavail}

In this scenario, the data generated by the injection and recovery artificial sources in order to meet the requirements to characterize the spuriousness parameter is made available so that users may calculate detection efficiencies. For example, a {\tt DIASource}-like catalog for the fake injected point sources, which users could bin by the $\vv{P}$ relevant to their science and generate $\eta(\vv{P})$. Since the current OSS requirements are to characterize the relationship between $\tau_{\mathcal{S}}$ and sample completeness {\it only as a function of visit image qualities} for {\tt DIASources} with SNR$>$5 (\S~\ref{sec:docs}), this scenario does not guarantee that these artificial sources will be adequate for all science use-cases. 

{\bf Scope --} Possible expansion of scope to make available the fake-source catalogs. \\
{\bf Risk --} Minor risk, if the fake sources do not adequately cover $\vv{P}$, for the same reason as in \S~\ref{ssec:opts_no}. \\
{\bf Requirements --} Does not violate or fulfill any requirements. \\
{\bf Science --} Allowing users to build detection efficiency matrices from the same fake sources as are used to characterize spuriousness would enable at least some scientific analyses.

% % % % % % % % % % % % % % % % 
\subsection{Ensure the Fakes Injected for Spuriousness Characterization Meet Science Goals}\label{ssec:opts_ensurefakeP}

This scenario is similar to that in \S~\ref{ssec:opts_makefakeavail}, except the fake sources that are injected and recovered in order to meet the requirements to characterize the spuriousness parameter are scientifically validated to cover the parameters needed for scientific analyses, $\vv{P}$, as listed in Table \ref{tab:eta_pars}.

{\bf Scope --} Minor expansion of scope to validate the artificial sources cover an adequate range of parameter space, $\vv{P}$, and to make available the fake-source catalogs. \\
{\bf Risk --} No risk. \\
{\bf Requirements --} Does not violate or fulfill any requirements. \\
{\bf Science --} Allowing users to build detection efficiency matrices from a scientifically-validated set of artificial sources would enable more scientific analyses.

% % % % % % % % % % % % % % % % 
\subsection{Generate and Provide Detection Efficiencies}\label{ssec:opts_deteffs}

This scenario takes that of \S~\ref{ssec:opts_ensurefakeP} one step further, and has the DMS generate and provide the detection efficiency matrix, $\eta(\vv{P})$, as a scientifically validated and verified data product.

{\bf Scope --} Moderate expansion of scope to generate $\eta(\vv{P})$. \\
{\bf Risk --} No risk. \\
{\bf Requirements --} Does not violate or fulfill any requirements. \\
{\bf Science --} Enables scientific analyses for all that need detection efficiencies.

% % % % % % % % % % % % % % % % 
%\subsection{Generate Detection Efficiencies Without Fake Injection}\label{ssec:opts_nofakes}
%\textcolor{red}{MLG: I've heard RL say there are other ways to generate detection efficiencies than fake injection, but outside of co-temporal data of superior quality (which will not be available), I'm not sure how to know how many real things are missed as a function of apparent magnitude and other parameters. Maybe RL can fill in this section.}

\subsection{Inject Fakes during Prompt or Data Release Processing?}\label{ssec:opts_fakeswhen}

Here is considered three possible points in the data processing where the fake injection could be performed: during Prompt processing (\S~\ref{sssec:opts_fakeswhen_PP}), on an intermediate timescale between Prompt and Data Release processing (\S~\ref{sssec:opts_fakeswhen_int}), and during DR processing (\S~\ref{sssec:opts_fakeswhen_DR}).

\subsubsection{During Prompt Processing}\label{sssec:opts_fakeswhen_PP}

Inject fake sources into the live data which is processed within 60 seconds of image readout ({\tt OTT1}). With this option, fake sources would be injected {\it on the fly into every new visit image from the telescope processed with DIA} (or into the template image as negatives). This may seem like an extreme option to propose, but as discussed in \S~\ref{ssec:sci_trans} some previous transient surveys have injected fakes into their real-time pipelines, so we consider it here as well. 

{\bf Scientific Motivation for Prompt Fakes --} Surveys that plant artificial sources into live data processing typically use realistic light curves that represent their target population (e.g., color, duration, and location), and inject the fakes into sequential images in order to simulate real transients. The objective of this level of real-time injection is usually not just to characterize the detection efficiency, but also biases in the survey's classification algorithms and/or follow-up programs. Simulating fake sources that represent {\it real transient light-curves} in sequential images and in different filters in a realistic way is {\it not} being proposed here. Furthermore, most of the scientific analyses that require detection efficiencies, such as occurrence rates and population studies, would be done with the DIA products from a data release, and not the prompt products. A continually-updated detection efficiency matrix, $\eta(\vv{P})$, that incorporates data from fakes injected during Prompt processing does not have a strong scientific motivation.

{\bf Interference with Astrophysical Sources -- } In this scenario, fake sources would be planted into new images in a manner that samples the range of parameters for $\eta(\vv{P})$. This process would be comprised of three steps: (1) identify coordinates where the fakes should be planted, (2) fake injection into the image, and (3) bookkeeping for the fakes to ensure they do not contaminate the Alert Stream or the {\tt DIASource} catalog. Fakes should not be injected at random locations because it is important to sample regions with higher surface brightness and to avoid the locations of known {\tt DIAObjects}. If 1000 fakes are  assigned random locations and injected into a 3.2 Gp image, and we assume that image has 10000 (randomly-distributed) true sources in it, then the probability that none of the fakes are coincident with one of the true sources is $0.9968$. However, over a full night of 1000 visits, the probability that none of the fakes ever landed on a true source in any image is $0.0437$, and it is most likely ($P=0.2218$) that 3 fakes would have interfered with true sources. 
% from scipy.stats import hypergeom
% import matplotlib.pyplot as plt
% [M,n,N] = [3200000000,10000,1000]
% rv = hypergeom(M,n,N)
% print( rv.pmf([0,1,2,3,4]) )
% [  9.96843454e-01   3.11521588e-03   4.86216368e-06   5.05362508e-09   3.93505558e-12]
% [M,n,N] = [3200000000000,10000000,1000000]
% rv = hypergeom(M,n,N)
% print( rv.pmf([0,1,2,3,4]) )
% [ 0.04367555  0.13880061  0.21497859  0.22180274  0.17274015]

{\bf Benefits and Drawbacks of Prompt Fakes from a DM Perspective -- } Two of the benefits (to LSST DM) of fake injection during Prompt processing are that (1) it would negate the need for separate re-processing of all fake-infused images, which could increase the total computational budget by up to a factor of $2$, and (2) it could offer real-time feedback on evolution in the survey's completeness or purity, which might be useful --- however, real-time feedback is not a necessary component of the DMS and could be obtained via the spuriousness parameters, as completeness and purity correlate mainly with bulk image properties. Two of the main drawbacks of planting fakes into "live" data are that (1) only a small number should be planted so as to minimize the risk of interference with real phenomena and (2) the additional steps of simulating, planting, and verifying fakes must be included in the computational budget for Prompt processing, which completes within $60$ seconds for every new direct image and is already tightly constrained. 

{\bf Scope --} An expansion of scope in terms of FTE work hours and computational resources. \\
{\bf Risk --} A risk to the DMS by adding three steps to the 60-second processing budget and potentially interfering with the completeness and purity of {\tt DIASources}. \\
{\bf Requirements --} Does not violate any requirements (and is not necessary to meet any requirements). \\
{\bf Science --} This option would provide scientifically useful detection efficiencies, however, it may compromise science results if it interferes with the completeness and purity of {\tt DIASources}. As there would be a limit on the number of fakes injected into every image, and restrictions that those fakes be away from most true transients and variables, this method would not provide the {\it best} characterization of the survey's detection efficiencies.

% % % % % % % % % % % % % % % % 
\subsubsection{On an Intermediate Timescale}\label{sssec:opts_fakeswhen_int}

As a compromise between injecting fakes during Prompt Processing (above) and during Data Release Processing (below), fakes could be injected and recovered on a intermediate timescale (e.g., daily, weekly, monthly). There would be no need to reprocess {\it every} image because the goal is to build up a detection efficiency model as a function of parameters like host background, seeing and airmass. This pipeline could include only images in the parameter space where additional fakes are required. However, as with the proposed option to do fake injection during Prompt processing, there is no science case (or internal use-case) that requires the detection efficiencies updated in real time.  

{\bf Scope --} An expansion of scope in terms of both FTE and computational resources of the DMS. \\
{\bf Risk --} A risk to the DMS (increasing the amount of processing done in between DRs). \\
{\bf Requirements --} Does not violate any requirements. \\
{\bf Science --} This option would enables rates analyses on the Prompt data products, but these analyses are more likely to be done on the DR data products anyway, so it is unlikely that this option opens the door for any new --- or otherwise inaccessible --- science. 

% % % % % % % % % % % % % % % % 
\subsubsection{During Data Release Processing}\label{sssec:opts_fakeswhen_DR}

The benefits of implanting artificial sources into images during the DIA which occurs as a part of the annual DR processing is that fakes can be injected (1) only in locations where there were no detected {\tt DIASources} and (2) in all images without increasing the total computational budget by any more than is required to inject the PSFs. As described in Section \ref{sec:docs}, it is likely that fake injection would be done as part of DIA during DR processing anyway, in order to characterize the spuriousness parameter. This option is only be adding a step to ensure that the fakes are injected in a way that samples the parameter space $\vv{P}$ (Table \ref{tab:eta_pars}), as needed to use the fakes for detection efficiencies. These DR-derived detection efficiencies could be used on the Prompt data products for the following year. 

{\bf Scope --} A mild expansion of scope in terms of FTE, but potentially not in terms of computational resources. \\
{\bf Risk --} No risks. \\
{\bf Requirements --} Does not violate any requirements. \\
{\bf Science --} Enables science with both the DR and Prompt data products. 


% % % % % % % % % % % % % % % % 
\subsection{Could Detection Efficiencies be Derived from Provided Data Products?}\label{ssec:docs_derDE}

Could the spuriousness parameter $\mathcal{S}$, and the relationship between $\tau_{\mathcal{S}}$ and completeness --- both of which are specified by the OSS to be included in the data products (\S~\ref{ssec:docs_oss}) -- be used to create a full detection efficiency matrix, $\eta(\vv{P})$? In other words, could the following process be used? (1) bin the {\tt DIASources} by $\vv{P}$; (2) calculate the mean spuriousness as a function of apparent magnitude in the bin ($\bar{\mathcal{S}}(m)$); (3) use the established relationship between $\tau_{\mathcal{S}}(m)$ and completeness (\S~\ref{ssec:docs_oss}) to derive $\eta(\vv{P})$.

No. First, the relation between $\tau_{\mathcal{S}}$ and completeness (\S~\ref{ssec:docs_oss}) provides the completeness for all sources with $\mathcal{S}>\tau_{\mathcal{S}}$, whereas the scientific analyses will want the completeness in discrete bins. Second, this relation has already marginalized over parameters $\vv{P}$, and to attempt to "reverse-engineer" them will not be accurate. Third, using the {\tt DIASources} to derive $\eta(\vv{P})$ will lead to a bias, since only {\it already detected} objects are contributing to the detection efficiency. 
% the current OSS requirements are to characterize the relationship between $\tau_{\mathcal{S}}$ and sample completeness {\it only as a function of visit image qualities} (\S~\ref{sec:docs}). 


% % % % % % % % % % % % % % % % 
\subsection{Summary of Options}\label{ssec:opts_sum}

It would be scientifically useful -- with only a potential small increase in scope, if any -- to ensure that the artificial sources implanted to characterize the {\tt DIASouce} spuriousness parameter sample the parameter space needed for scientific analyses involving detection efficiencies, $\vv{P}$ (e.g., Table \ref{tab:eta_pars}), and to make the data from the injection and recovery of fake sources available to users so that they can build detection efficiency matrices $\eta(\vv{P})$. It would be even more useful to provide $\eta(\vv{P})$ as a scientifically validated data product. For both scenarios, doing the fake injection during the DIA which occurs as a prt of the annual Data Release processing both achieves the science goals and minimizes scope increase and risk to the DMS.


\section{Options for Fake Point Source Injection Techniques}\label{sec:tech}

There are a variety of ways in which fake sources can be simulated and injected into the images. Some options are more suitable for transients (\S~\ref{ssec:tech_new}), and some for variable stars (\S~\ref{ssec:tech_pre}). The following is a precursory presentation of the options, some of which have been used in the science studies discussed in \S~\ref{sec:sci}. Typically, artificial sources -- positive or negative -- are added to the direct image {\it before} that image enters the difference imaging pipeline, so that the detection efficiency captures the end-to-end pipeline efficiency for detecting difference-image sources. (This is why fakes are not typically injected into the final difference image prior to source detection).

% % % % % % % % % % % % % % % % 
\subsection{Simulating New Fake Objects}\label{ssec:tech_new}

This applies to point sources that appear where there was no point source in the template image, such as transients like supernovae, variable stars that are undetectable in their quiescent state, and moving objects (assuming they're slow-moving enough to not appear as trailed sources, which is a different problem not included in this study). 

Artificial sources that represent new fake objects would be planted in and around galaxies in a way that samples the environments of known transients (and serves the use-cases of variable stars and moving objects projected on background galaxies), would adequately sample areas of open space where most moving objects, some variable stars, and high-$z$ transients with undetectable hosts will be found, and also in crowded fields where many variable stars will be discovered.

{\bf Model PSF ---} A 2D model for the PSF is added to the direct image in order to simulate a new point source. The shape of the PSF is derived from known trends with, e.g., the focal plane location or airmass (DCR), and the brighter/fatter effect.

{\bf Clone-Stamping ---} A nearby star is cut out and rescaled, and used as the simulated point source. One of the main drawbacks of using clone-stamping with LSST images is that incorporating the brighter/fatter effect into the simulation requires either that a star which is both nearby and of a similar brightness be used or that a model component added to the clone star to appropriately change the shape for the simulated brightness. Another drawback of clone-stamping is that very sparse/crowded fields might not have enough nearby/isolated point sources to use.

To decide between model PSFs and clone-stamping will require some testing in order to properly assess their performance and load on the computational resources. However, since knowing the PSF very accurately is something the LSST DMS will already be doing, it seems likely that the Model PSF option should be easier.

% % % % % % % % % % % % % % % % 
\subsection{Simulating Variability in Real Objects}\label{ssec:tech_pre}

This applies to point sources that appear in both the template and direct image, such as variable stars and AGN. In this case, artificial variability is added to an existing point source in the direct image. Extra steps would need to be taken to ensure any real, low-level variability does not affect the results.

{\bf Model PSF Variable Component ---} A 2D model for the PSF with the desired flux of the variable component of the source is added to the object's 2D profile. This might only be useful for probing small flux changes, as it would be difficult be consistent with the brighter/fatter effect.

{\bf Scaling-in-Place ---} Cutout the star, multiply its 2D profile by a scalar in order to make it brighter or fainter, and add it back to the image. This could be modified to account for the brighter/fatter effect by, e.g., convolving with a kernel that both applies the effect and the desired variability, instead of multiplying by a scalar.

\subsubsection{Planting in a Template Image}\label{ssec:tech_pre_temp}

Could there be situations in which, in order to simulate variability, adding a new point source to the template only, or to both the template and the direct image, is needed? For example, in crowded fields, the detection efficiency for variable components of stars that are faint in the template image might be difficult to accurately measure because faint stars are hard to detect and isolate in crowded fields. This a necessary step in applying either of the two above methods for injecting artificial variability in real objects. Thus, it might be necessary to simulate new faint stars in the template {\it and} the direct image, with some flux difference between them, in order to derive detection efficiencies that are not dominated by the brighter, more well-distinguished stars in a crowded field. This would add complexity to the issue and might further expand the scope of the proposed option to provide scientifically validated injected fakes. 

% % % % % % % % % % % % % % % % 
\section{Summary, Open Questions, and Suggested Future Work}\label{sec:future}

\begin{figure}
\begin{center}
\includegraphics[width=15cm,trim={0cm 0cm 0cm 0cm}, clip]{figures/option_matrix.png}
\caption{A summary of the options with evaluated criteria, based on \S~\ref{sec:opts}. \label{fig:options}}
\end{center}
\end{figure}

This study is not yet finished and there remain some open questions to address, and further work is likely needed in order for an informed decision about the options proposed.

{\bf Open Questions for DM-SST:}\\
(1) Have DM's plans evolved away from what's in the documents (\S~\ref{sec:docs})? \\
(2) Does this work constitute a DMTN? It is not very technical -- yet. One option might be to take this to the TVS community for input and write a joint TVS-DM document, which incorporates the further study items below.

{\bf Further Study:}\\
 - establish the extent of the parameter space, $\vv{P}$\\
 - evaluate the accuracy needed for $\eta(\vv{P})$\\
 - does the mode of fake injection (\S~\ref{sec:opts}) matter for science?\\
 - will LSST's DIA be good enough to simply inject fakes into difference images?\\

\end{document}

